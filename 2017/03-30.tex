\documentclass[12pt,a4paper]{article}
\usepackage[T1]{fontenc}
\usepackage[swedish]{babel}
\usepackage[utf8]{inputenc}
\usepackage{listings}
\usepackage{graphicx}
\usepackage{color}

\definecolor{green}{rgb}{0,0.6,0}
\definecolor{darkgreen}{rgb}{0.1,0.3,0.1}
\definecolor{mymauve}{rgb}{0.58,0,0.82}

\lstset{ %
  backgroundcolor=\color{white},   % choose the background color; you must add \usepackage{color} or \usepackage{xcolor}
  basicstyle=\footnotesize,        % the size of the fonts that are used for the code
  breakatwhitespace=false,         % sets if automatic breaks should only happen at whitespace
  breaklines=true,                 % sets automatic line breaking
  captionpos=b,                    % sets the caption-position to bottom
  commentstyle=\color{mygreen},    % comment style
  deletekeywords={...},            % if you want to delete keywords from the given language
  escapeinside={\%*}{*)},          % if you want to add LaTeX within your code
  extendedchars=false,              % lets you use non-ASCII characters; for 8-bits encodings only, does not work with UTF-8
  frame=single,                    % adds a frame around the code
  keepspaces=true,                 % keeps spaces in text, useful for keeping indentation of code (possibly needs columns=flexible)
  keywordstyle=\color{blue},       % keyword style
  language=Octave,                 % the language of the code
  morekeywords={*,...},            % if you want to add more keywords to the set
  numbers=left,                    % where to put the line-numbers; possible values are (none, left, right)
  numbersep=5pt,                   % how far the line-numbers are from the code
  numberstyle=\tiny\color{mygray}, % the style that is used for the line-numbers
  rulecolor=\color{black},         % if not set, the frame-color may be changed on line-breaks within not-black text (e.g. comments (green here))
  showspaces=false,                % show spaces everywhere adding particular underscores; it overrides 'showstringspaces'
  showstringspaces=false,          % underline spaces within strings only
  showtabs=false,                  % show tabs within strings adding particular underscores
  stepnumber=1,                    % the step between two line-numbers. If it's 1, each line will be numbered
  stringstyle=\color{mymauve},     % string literal style
  tabsize=2,                       % sets default tabsize to 2 spaces
  title=\lstname                   % show the filename of files included with \lstinputlisting; also try caption instead of title
}


\begin{document}
	% Change to "Årsmöte" or "Annual Meeting" if this is held, or "Extra årsmöte" or "Extra annual meeting".
	\title{\Huge Protokoll För Styrelsemöte Sportsektionen}
	\date{30 mars 2017}
	\maketitle

	\null
	\vfill

	\clearpage

	\begin{enumerate}

		\item Mötets öppnande
		
			Mötet öppnades 12:00.
			
		\item Val av mötessekreterare
		
			Niklas Åkerlund är mötessekreterare.
			
		\item Val av mötesordförande
		
			Marko Saukko är mötesordförande.
			
		\item Val av justerare tillika rösträknare
		
			Cristoffer Lagergren och Anton Österberg är justerare tillika rösträknare.
			
		\item Fråga om mötets stadgeenliga utlysande
		
			Mötet stadgeenligt utlyst.
			
		\item Justering av röstlängd
		
			\begin{itemize}
				\item Anton Österberg
				\item Cristoffer Lagergren
				\item Fredrik Junghem
				\item Marko Saukko
				\item Max Bertilsson
				\item Niklas Åkerlund
			\end{itemize}
				
		\item Eventuella adjungeringar
		
			Inga adjungeringar.
			
		\item Fastställande av fördragningslista
		
			Utskickad mötesagenda fastställd med en extra punkt om adminrättigheter.
		
		\item Rapporter från våra aktiviteter
		
		    \begin{enumerate}
				\item Badminton
				    
				    De nya internationella studenterna är inte längre med, de spelar istället i Frescatihallen då det är närmare till deras boenden och billigare. Priserna för att hyra banorna har blivit dyrare, däremot drivs aktiviteten med ett plusresultat om 20:-/tillfälle.
				    
				\item Basket
				    
				    Basketens sista inplanerade inomhuspass är den 2 april 2017, därefter blir det eventuellt utomhuspass varje vecka. Aktiviteten plågas av ett nedåtgående deltagande. Fortsatta inomhuspass ska inte bokas om inte ett tydligt intresse finns.
				    
				\item Fotboll
				    
				    Fotbollen går fortsatt bra, passet lördag 1 april blev fullbokat några timmar efter inbjudan hade skickats ut. Man ser gärna att det skulle spelas utomhus, men att det blir konsekvenser som att hitta en ledig fotbollsplan och att det kanske inte passar alla som är med lika bra som inomhusfotbollen. I mån av intresse och tid kan event bokas utanför den vanliga tiden.
				    
				\item Innebandy
				    
				    Innebandyn behöver få besked om vart passen skall hållas i höst då priset för att hyra hallen vid Tegelhagsskolan kan komma att ändras från 80:-/timme till ca 500:-/timme. Innebandyn ska spika ett datum för när den årliga matchen mot KTH skall spelas, i samband med matchen skall det gärna ordnas ett evenemang i samarbete med KM så att en pub är öppen efter matchen (troligtvis en onsdag).
				    
				\item Generellt
				    
				    Priset för hallen vid Kista Brandstation är en nackdel i jämförelse med hallen vid Tegelhagsskolan. Däremot är den bättre belägen än vad hallen vid Tegelhagsskolan är. En ny hall att hålla aktiviteterna (framförallt innebandyn) i (i Kista) behöver utforskas, en möjlighet är Actic. Priset på Actic kan vara högt vilket är en nackdel, kan eventuellt finnas studentpris. 
				    
				    Inbetalningar har fungerat bra för samtliga aktiviteter.   
				    
			\end{enumerate}
			
		\item SkiWeek 2018
			
			Cristoffer kommer inte att kunna anordna SkiWeek 2018, därför behövs en ny (eller två) ansvarig(a) hittas. Martin Murén och Oscar Thörnkvist skall vara intresserade. Som diskuterades vid förra mötet kommer Cristoffer att ta fram en överlämning om SkiWeek.
			
		\item Planering framåt
		    
		    \begin{enumerate}
				
				\item Ansvar och ambitioner
				    
				    Det skulle vara intressant att hitta en sektionssekreterare. En PR-ansvarig står högst upp på önskelistan. Vi behöver tydliggöra rollfördelningen inom sektionen. Vi vill nå ut till fler studenter, potentialen finns, men hur skall vi verkligen få folk att delta i aktiviteterna. 
				    
				\item Samarbete och öppenhet
				    
				    Skall hållas ett planeringsmöte med DISK där det planeras att finnas en grupp med medlemmar från varje sektion för att bygga samarbeten mellan sektionerna och promote:a de olika sektionerna. Marko, Max och Fredrik kommer att delta på planeringsmötet för att höra mer om det. DISK har tydligare riktlinjer om hur de olika sektionerna kan nå ut till studenterna, bland annat genom Skitviktigt, skärmarna till grupprummen och skärmarna inne i Foobar.
				    
				    Sportsektionen som sektion känns för isolerad, vi behöver investera mer tid till att bygga upp gemenskapen inom sektionen mer. Hur skall vi bli mer motiverade att hjälpas åt (t.ex. hitta ny ansvarig till innebandyn)? Många känner sig bekväma med att ansvara för sina aktiviteter och sedan är det inte mer med det.
				    
				\item "Vilka är sporten?"
				    
				    Vilka värderingar har Sportsektionen? Vilken kultur eftersträvar Sportsektionen? Vilken historia har Sportsektionen (har tagits upp tidigare och information har tagits fram). Vad erbjuder Sportsektionen och varför ska man vara delaktig i Sportsektionen? Vi behöver integrera detta i arbetet med Insparken. Mer om detta i sektionen för uppföljning. Vi ska bli bättre på att promote:a för de som deltar kontinuerligt i aktiviteterna att gå med som aktiva i Sportsektionen.
				    
				\item Vad vill vi ge till dem som tar över?
				    
				    Behöver diskuteras mer innan sommaren då minst tre (Anton, Cristoffer, Niklas) av oss försvinner, troligtvis fyra (även Anna) stycken. Hjälpmedel så som beskrivningar/guider för att hantera webbsidan och aktiviteter.
				    
				\item Representera sektionen
				    
				    Vi behöver fundera på vad vi ska göra för att synas. Presentera Sportsektionen inför förstaårs studenter på föreläsningar i början av hösten 2017. Köpa varsin t-shirt eller piké som kan användas under Insparken och andra evenemang.
				    
			\end{enumerate}
		    
		\item Uppföljning
		    
		    Från förra mötet:
		    \begin{enumerate}
		        
				\item Webbsidan arbetas fortsatt med av Anton och Anna. Niklas kan tänkas hjälpa till under hösten.
				
				\item Betalningar sköts bättre nu efter att Sebastian från DISK Styrelse var med och förklarade för oss hur det skall gå till.
				
				\item Vi diskuterade fortsatt Insparken 2017. Planen är att vara representativa varje dag första veckan och åtminstone fredag vecka två.
					
			\end{enumerate}
			
			Inför nästa möte:
			\begin{enumerate}
				
				\item Samtliga skall ta fram två idéer på aktiviteter som kan genomföras under Insparken.
			    
				\item Planera in de föreslagna aktiviteterna för Insparken.
				
				\item Insparkstidningen behöver en text.
				
				\item Ta fram hur vi på ett bra sätt kan promote:a de veckovisa aktiviteterna under Insparken.
				
				\item Vad skall sägas under vallningsdagen?
				
				\item Ta fram små lappar som kan delas ut under vallningsdagen och Insparken om Sportsektionen.
				
				\item Vi ska lära oss av Närs och hur de lockar till sig nytt folk. Vad gör dom så bra?
					
			\end{enumerate}
		    
		\item Övriga under mötet uppväckta frågor
			\begin{enumerate}
			
				\item Adminrättigheter i grupper, se till att de som har adminrättigheter men inte ska ha det lämnar ifrån sig dessa, eller om möjligt ta bort rättigheterna.
					
			\end{enumerate}
			
		\item Armhävningar
		
		    Fem armhävningar per uppäten munk utförda.
		
		\item Mötets avslutande
		
		    Mötet avslutades ca 14:20.
		
	\end{enumerate}
\end{document}